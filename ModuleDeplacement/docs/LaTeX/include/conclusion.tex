% Ameliorations possibles
%\section{Améliorations possibles}

% Conclusion à proprement parler
\section{Conclusion}
Ce projet est pour nous une très bonne expérience personnelle et pédagogique.\\
Au yeux du cahier des charges initial, nous avons plutôt bien réussi notre travail, mais vis à vis de nos ambitions nous sommes un peu en deçà des objectifs prévus pour la date à laquelle ce rapport est rédigé.\\

Nous avons rencontré quelques difficultés, en partie dues au travail sur FPGA qui est une chose presque nouvelle pour nous. Ce mode de raisonnement n'est pas celui auquel nous sommes habitué et il nous a fallu un certains temps d'adaptation avant de commencer à se sentir plus à l'aise dans le développement VHDL.\\
Ça aura été finalement très enrichissant, puisque nous avons en fait découvert un tout nouveau domaine. Il nous aura été difficile d'accès, mais nous devons reconna\^itre que nous sommes maintenant séduits par les possibilités qu'apportent le travail sur un FPGA. Et ce nouveau raisonnement qui nous paraissait un peu rebutant au départ est, une fois qu'on l'a apprivoisé, très riche et plein de bonnes surprises.\\
Quand on développe sur des circuits logique programmables, la réflexion faite en amont, bien avant de rédiger une seule ligne de code, est la clé de la réussite. Cette réflexion, ce raisonnement fera absolument toute la différence. Et c'est là où cela diffère des nos habitudes de développeurs logiciels, nous étions habitué à pouvoir coder immédiatement et réfléchir/ajuster pendant. Ce n'est pas possible ici.\\

Nous avons aussi rencontré beaucoup de difficultés sur la gestion du temps. Habitués à pouvoir fournir du travail supplémentaire pour perfectionner nos projets, nous n'avons pas été assez vigilants. Étant novices dans le domaine, nous avons sûrement plus subit les aléas  que dans nos projets précédents. Notre emploi du temps et nos autres investissements cette année ne nous ont pas laissé la latitude nécessaire pour mener notre projet comme nous l'aurions voulu, mais c'est une belle leçon que nous pouvons en tirer. On n'est pas toujours ma\^itre du temps à notre disposition, ce qui compte c'est d'en \^etre conscient et d'agir en conséquence.\\


Nous ne nous arr\^eterons pas à ce stade en ce qui concerne le développement de notre application. Nous avons la ferme intention d'avoir un système complet et facilement réutilisable, gérant l'intégralité du déplacement d'un robot mobile. Et nous avons de forts espoirs de voir notre travail faire ses preuves à la Coupe de France 2014 qui aura lieu la semaine du 30 mai.\\