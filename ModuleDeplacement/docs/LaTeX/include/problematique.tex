\subsection{Problématique}
Notre projet s'inscrit dans le cadre du club Robotech Lille, club de robotique de l'école Polytech Lille, et de la Coupe de France de Robotique organisée par Planètes Sciences.\\

Cette Coupe de Robotique exige la conception et la réalisation de robots mobiles autonomes.\\
A Robotech Lille, les informations propre au déplacement sont données par ces fameux encodeurs à signaux à quadratures de phases.\\
Il est donc nécessaire que soit intégrée au robot un système de décodage des signaux à quadratures de phase.\\

Précédemment, le robot était géré par un unique micro-contrôleur, aussi en charge du décodage de ces signaux.\\
Cette année est née la volonté d'attribuer la gestion complète du déplacement du robot à un sous-module, facilement réutilisable et indépendant de la structure du robot. Il a été décidé, pour des raisons que nous allons voir ensuite, que ce sous-module allait \^etre géré par un FPGA\footnote{Field-Programmable Gate Array, un réseau de portes programmables (Voir 
\href{http://fr.wikipedia.org/wiki/Circuit_logique_programmable}{Wikipedia : Circuits Programmables})}. La gestion du décodage des signaux en quadrature de phase doit donc \^etre elle aussi assumée par le FPGA, c'est à cette partie là que nous nous intéressons.\\