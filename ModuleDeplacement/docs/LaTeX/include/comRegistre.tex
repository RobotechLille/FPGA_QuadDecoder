\section {Module Registre}
Afin de pouvoir réaliser l'asservissement des moteurs, des données doivent être enregistrées sur une mémoire qui sera accessible par le module d'asservissement 
et par un module qui traitera les informations reçues sur le port série. 

Cette mémoire sera faite au moyen d'un gros registre découpé en différentes données(comme la valeur des correcteurs, des encodeurs, de l'odométrie...) . 
Le module d'asservissement aura un accès direct au registre sans passerelle, mais les informations reçues par le port série sera traité dans un module à part.
Ce module va ``écouter'' les données sur le port série et fera les actions necessaires. Ces actions pourront être la lecture d'une donnée du registre mais aussi 
l'écriture afin de changer des valeurs telle que demander au robot d'aller plus vite ou moins vite ou de tourner d'un angle particulier.


Afin de traiter correctement les informations reçues, nous avons mis en place le protocole suivant :

\begin{itemize}

\item Premier octet, initialisation : 0x00
\item Deuxième octet, ordre :
  \begin{itemize} 
    \item 0x01 : lecture
    \item 0x02 : écriture
   \end{itemize}
\item Troisième octet, adresse : entre 0x00 et 0x17 (24 adresses possibles)
\item Quatrième octet, données : entre 0x00 et 0xFF (en cas de lecture 0x55)
\item Cinquième octet, fin : 0xFF
\end{itemize}


Le FPGA n'aura jamais à demander d'informations au Raspberry par conséquent, lorsqu'il y aura une demande de lecture, le FPGA renverra directement la valeur demandée sans utilisation de protocole particulier.
