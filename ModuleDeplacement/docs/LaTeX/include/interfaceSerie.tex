\section{Interface série}
Dans le but d'établir une communication entre le Raspberry (partie intelligente) et le FPGA nous avons adopter un mode de communication série. Ce protocole étant relativement simple à 
coder et à mettre en place, c'est pourquoi nous avons décider de l'utiliser au sein du robot pour l'échange de données.

\subsection{Reception}
Ce programme agit sur chaque front montant de l'horloge. Son but sera de permettre la réception des
données à 9600 bauds(debit paramétrable par une variable générique).
A chaque front montant de l'horloge,nous incrémentons un compteur
de 1 jusqu'à compter la période d'un bit (5208 dans le cas d'une transmission à 9600 bauds avec un horloge de 50MHz). Lorsque ce chiffre est atteint nous récupérons la
valeur du bit Rx et la stockons dans la variable data\_next par décalage à chaque fois. De cette
manière nous aurons, dans data\_next, la valeur des bits du message mais dans le sens inverse. A
chaque récupération de bit, la variable i est incrémentée de afin de connaître la fin de réception de
la donnée.
Au premier front d'horloge (bit de start à 1), le compteur est initialisé à sa moitié afin que la
récupération des données se fasse au milieu de la période du bit pour être sur d'avoir la bonne
valeur.
Lorsque i=nombre bit de données + 2 (le nombre de bit de données est paramétrable par une variable générique), tous les bits du messages ont été enregistrés.
Nous écrivons donc la donnée ainsi reçue sur le registre de sortie de manière inversée (oData(0)=Data\_next(8) ...).

\subsection{Emission}
Ce programme, sur chaque front montant va commencer la transmission si le bit iEnableTransmit est à l'état haud. A ce moment là on va incrémenter un compteur qui
permettra de faire une communication à un debit donné. C'est à dire que pour une communication à 9600 baud par exemple, on doit laisser le bit de data à son état 
pendant 5208 coups d'horloge avec un horloge à 50MHz. De plus, ce module établie une communication série avec un bit de start (passage de 1 à 0), un nombre de bit de donné
paramétrable (doit être le même en emission et réception) et un bit de stop.
On avons donc une communication très simple, sans détection/correction d'erreurs.

\subsection{Module de gestion}
Le module de gestion permettra d'instancier les deux composants précédemment créés et de faire un mappage complet des entrées sorties. Ainsi nous n'aurons plus qu'un module
général où toutes les entrées sorties seront déclarées. De même, c'est dans ce fichier où il suffira de changer les différents paramètres génériques pour configurer la communication
(fréquence d'horloge, débit de communication, nombre de bit de données).
